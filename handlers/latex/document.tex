% SPDX-FileCopyrightText: 2025 Steffen Vogel <post@steffenvogel.de>
% SPDX-License-Identifier: Apache-2.0

\documentclass{article}
\usepackage{amsmath}
\usepackage{listings}
\usepackage{xcolor}
\usepackage[a4paper, total={6in, 8in}]{geometry}
\geometry{margin=1in}

\definecolor{stringColor}{HTML}{718a62}
\lstdefinelanguage{Nix}{
	keywords=[1]{true, false, null},
	keywords=[2]{let, in, with, rec, inherit},
	keywords=[3]{toString},
	keywordstyle=\color{blue},
	sensitive=true,
	% Comments
	comment=[l]{\#},
	% Strings
    morestring=*[d]{"},
    morestring=[s][\color{red}]{\$\{}{\}},
    morestring=*[d]{''},
	stringstyle=\color{stringColor},
}

\lstset{
basicstyle=\small\ttfamily,
columns=flexible,
breaklines=true
}

\title{Nixpresso \LaTeX{} Example}
\author{Steffen Vogel}
\date{\today}

\begin{document}
\maketitle
\section{Introduction}
Hello world! This PDF document is rendered on-demand by Nixpresso.

\section{Maxwell Equations}
``Maxwell's equations'' are named for James Clark Maxwell and are as follow:
\begin{align}             
\vec{\nabla} \cdot \vec{E} \quad &=\quad\frac{\rho}{\epsilon_0} &&\text{Gauss's Law} \\      
\vec{\nabla} \cdot \vec{B} \quad &=\quad 0 &&\text{Gauss's Law for Magnetism}\\
\vec{\nabla} \times \vec{E} \quad &=\hspace{10pt}-\frac{\partial{\vec{B}}}{\partial{t}} &&\text{Faraday's Law of Induction} \\ 
\vec{\nabla} \times \vec{B} \quad &=\quad \mu_0\left( \epsilon_0\frac{\partial{\vec{E}}}{\partial{t}}+\vec{J}\right) &&\text{Ampere's Circuital Law}
\end{align}

\section{Request}

\begin{lstlisting}[language=Nix]
@prettyRequest@
\end{lstlisting}

\end{document}